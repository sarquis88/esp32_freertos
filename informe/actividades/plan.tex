\documentclass{article}

\usepackage[spanish]{babel}
\usepackage{titling}
\usepackage{graphicx}
\usepackage{graphicx}
\usepackage[export]{adjustbox}
\usepackage{hyperref}
\usepackage{ragged2e}
\usepackage{indentfirst}
\usepackage{float}
\usepackage{microtype}
\usepackage[bottom]{footmisc}
\usepackage{fancyhdr}
\fancyhead[R]{2020}\fancyhead[L]{UNC - FCEFyN} \fancyfoot[C]{\thepage}
\pagestyle{fancy}

%%%%%%%%%%%%%%%%%%%%%%%%%%%%%%%%%%%%%%%%%%%%%%%%%%%%%%%%%%%%%%%%%%%%%%%%%%%%%%%

\begin{document}

    \begin{center}
        {\huge{PLAN DE ACTIVIDADES}}
    \end{center}

    \textbf{OBJETIVO} \par
    El objetivo de la práctica supervisada es implementar un \emph{software}
    que, corriendo en el embebido ESP32, pueda monitorear el movimiento de 
    animales vacunos. \newline

    \textbf{PLAN}
    \begin{itemize}
        \item Investigación sobre el microcontrolador ESP32.
        \item Instalación del sistema operativo FreeRTOS.
        \item Pruebas varias sobre el \emph{hardware}. Familiarización.
        \item Integración del sensor de aceleración.
        \item Implementar los requerimientos del sistema
        \item Hacer pruebas. Analizar resultados.
        \item Corregir implementación, de acuerdo a los resultados obtenidos.
        \item Documentar.
    \end{itemize} 

    \textbf{DISTRIBUCIÓN SEMANAL DE LA CARGA HORARIA} \par
    La presente práctica profesional supervisada se llevará a cabo cinco días
    de la semana, en una franja de cuatro horas diarias como se indica en la 
    siguiente tabla:
    
    \begin{table}[h]
        \centering
        \begin{tabular}{||c|c|c|c|c|} 
            \hline
            Lunes & Martes & Miércoles & Jueves & Viernes \\ [0.5ex] 
            \hline\hline
            9:00 - 12:00 & 9:00 - 12:00 & 9:00 - 12:00 & 9:00 - 12:00 & 9:00 - 12:00 \\
            \hline
        \end{tabular}
        \caption{Distribución de la carga horaria}
    \end{table}

\end{document}